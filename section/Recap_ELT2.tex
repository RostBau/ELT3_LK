\kommentar{Repetition ELT2} %Thema

\begin{karte}{Wie gross ist das Elektrische Feld, auf Kopfhöhe 2m) unter einer Hochspannungsleitung?\\
	\small höhe $h = 10m$, Leitungsdurchmesser $d = 3cm$, $U=100kV$}
	\small
	\begin{compactenum}
		\item Ladung berechnen:
		\begin{align*}
			U &= \int_{d/2}^{h} E \cdot dl = \int_{d/2}^{h} \frac{Q}{2 \pi \varepsilon_0 r l} \cdot dl  \\
			&=\frac{Q}{2 \pi \varepsilon_0 l} \cdot \left[ln(r)\right]_{d/2}^{h}\\
			Q & = \frac{U \cdot 2\pi \varepsilon_0\cdot l}{ln(2h/d)}
		\end{align*}
		\item In Formel von $E$ einsetzen. 
		\renewcommand\CancelColor{\color{red}}
		\begin{align*}	
			E &= \frac{Q}{2\pi \varepsilon_0\cdot r \cdot l} =
			\frac{1}{\cancel{2\pi \varepsilon_0}\cdot r \cdot \bcancel{l}} \cdot
			\frac{U \cdot \cancel{2\pi \varepsilon_0}\cdot \bcancel{l}}{ln(2h/d)}\\
			& = \left.\frac{U}{ln(2h/d)\cdot r}\right|_{r=8m} \hspace{-7mm} = 1.92kV/m
		\end{align*}
	\end{compactenum}
\end{karte}