\kommentar{Wellen \& Strahlungsfelder}  %Thema

\begin{karte}{Was beschreibt das Poynting-Theorem?}
	Grundsätzlich handelt sich beim Poynting-Theorem um Energieerhaltung mit Leistungsübertragung.
	
	\begin{equation*}
	\underbrace{-\frac{d}{d t} \int_{V}\left(w_{e}+w_{m}\right) d v}_{\text{ Zeitliche Änderung der Energie }}
	=
	\underbrace{\int_{V} p d v}_{\text{Leistung}}
	+
	\underbrace{\int_{A=\partial V} \vec{S} \cdot d \vec{s}}_{Leistungsübertragung}
	\end{equation*}
	
	Die Zeitliche Änderung der Energie in einem Volumen, sei sie nun zugeführt oder abgeführt worden, muss gleich gross sein wie die Abgegebene Leistung (Wärme, bei zugeführte Leistung bspw. Solarzelle) plus der austretende oder eintretende Leistung des Volumens.
\end{karte}

\begin{karte}{Was bezeichnet der Poynting Vektor?}
	\begin{equation*}
	\vec{S}=\vec{E} \times \vec{H}
	\end{equation*}
	
	Der Poynting Vektor ist die Leistungsdichte einer Fläche und hat somit die Einheit $[S] = \dfrac{W}{m^2}$.\\
	Der Vektor zeig in die Richtung der Leistung.\\[5pt]
\end{karte}