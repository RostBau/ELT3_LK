%
%    Autor: Simon Walker
%			simon.walker@hsr.ch
%	Lizenz: CC BY-NC-SA
%			Für bilder gelten eventuel andere Lizenzen dise sind jeweils dann als Kommentar hinter dem Include angegeben


%	 Dieses LaTeX Dokument benutzt eine Karteikartenklasse welche von Ronny Bergmann <mail@rbergmann.info> entwickelt wurde version 1.8b
%	 Die Dokumentation und die Karteikartenklasse sind unter folgendem Link erhältlich
%    https://github.com/kellertuer/Kartei

%------------------------------------------------------------
% Lernkarten zum Fach ELT3,
%------------------------------------------------------------

\documentclass[a7paper,10pt,grid=none% %Grösse der Karteikarten auswählen (A5-A9)
%,toc			%Komentarlöschen falls eine Zusammenstellung gewünscht wird
%,print			%Komentar löschen falls eine Ausdruckversion erstellt werden soll
]{kartei}

\usepackage[utf8]{inputenc} %UTF8
\usepackage{hyperref}

\usepackage{amsmath}
\usepackage{bm}
\usepackage{paralist} % für compactenum and compactitem

\usepackage[T1]{fontenc}% wichtig für Trennung von Wörtern mit Umlauten
\usepackage[ngerman]{babel}% deutsche Trennregeln
\usepackage{microtype}% verbesserter Randausgleich

\usepackage{hyphsubst}
\usepackage{lmodern} %Schriftart

\usepackage{tikz}
\usepackage{xstring} %If statements
\usepackage{scalefnt}

\usepackage{cancel} %\cancel befehl kürzen %Farbe ändern mit \renewcommand\CancelColor{\color{red}}


\usepackage{pgfplots}
\usepgfplotslibrary{fillbetween}

%Autor: Simon Walker
%Version: 1.1
%Datum: 09.11.2019


%Horizontaler Widerstand Parameter: xKoordinate, yKoordinate, Bezeichnung, Position (a=>above, b=>below)
%Anschluss ist  bei (x-0.7, y) & (x+0.7, y)
\newcommand{\tzRH}[4]{
	\draw [thick] (#1-0.7, #2-0.25) rectangle (#1+0.7, #2+0.25);
	\IfEqCase{#4}{
		{a}{\node [above] at (#1, #2+0.25) {\Large #3};}%
		{b}{\node [below] at (#1, #2-0.25) {\Large #3};}%
	}
}

%Vertikaler Widerstand Parameter: xKoordinate, yKoordinate, Bezeichnung, Position (r=>right, l=>left)
%Anschluss ist  bei (x, y+0.7) & (x, y-0.7)
\newcommand{\tzRV}[4]{
	\draw [thick] (#1-0.25, #2-0.7) rectangle (#1+0.25, #2+0.7);
	\IfEqCase{#4}{
		{r}{\node [right] at (#1+0.25, #2) {\Large #3};}%
		{l}{\node [left] at (#1-0.25, #2) {\Large #3};}%
	}
}

%Horizontaler Kondensator Parameter: xKoordinate, yKoordinate, Bezeichnung, Position (a=>above, b=>below)
%Anschluss ist  bei (x-0.25, y) & (x+0.25, y)
\newcommand{\tzCH}[4]{
	\draw [ultra thick] (#1+0.25, #2-0.6) -- (#1+0.25, #2+0.6);
	\draw [ultra thick] (#1-0.25, #2-0.6) -- (#1-0.25, #2+0.6);
	\IfEqCase{#4}{
		{a}{\node [above] at (#1, #2+0.5) {\Large #3};}%
		{b}{\node [below] at (#1, #2-0.5) {\Large #3};}%
	}
}

%Vertikaler Kondensator Parameter: xKoordinate, yKoordinate, Bezeichnung, Position (r=>right, l=>left)
%Anschluss ist  bei (x, y+0.25) & (x, y-0.25)
\newcommand{\tzCV}[4]{
	\draw [ultra thick] (#1-0.6, #2+0.25) -- (#1+0.6, #2+0.25);
	\draw [ultra thick] (#1-0.6, #2-0.25) -- (#1+0.6, #2-0.25);
	\IfEqCase{#4}{
		{r}{\node [right] at (#1+0.5, #2) {\Large #3};}%
		{l}{\node [left] at (#1-0.5, #2) {\Large #3};}%
	}
}

%Vertikale Spannungsquelle Parameter: xKoordinate, yKoordinate, Bezeichnung, Position (r=>right, l=>left), Ausrichtung(+=> + oben, -=> - oben)
%Anschluss ist  bei (x, y+0.5) & (x, y-0.5)
\newcommand{\tzUV}[5]{
	\draw [thick] (#1, #2) circle [radius=0.5];
	\IfEqCase{#5}{
		{+}{\node at (#1, #2+0.25) {\Large $+$};
			\node at (#1, #2-0.25) {\Large $-$};}
		{-}{\node at (#1, #2+0.25) {\Large $-$};
			\node at (#1, #2-0.25) {\Large $+$};}
	}
	
	\IfEqCase{#4}{
		{r}{\node [right] at (#1+0.5, #2) {\Large #3};}%
		{l}{\node [left] at (#1-0.5, #2) {\Large #3};}%
	}
}

%Knoten Parameter: xKoordinate, yKoordinate
\newcommand{\tzKN}[2]{
	\draw[fill=black] (#1, #2) circle(0.07);
}

%Schalter (Schliesser) Horizontal Parameter: xKoordinate, yKoordinate
%Anschluss ist bei (x-0.6, y) & (x+0.6, y)
\newcommand{\tzSH}[2]{
	\draw (#1-0.5, #2) circle(0.1);
	\draw (#1+0.5, #2) circle(0.1);
	
	\draw [thick] (#1-0.4, #2) -- (#1+0.5, #2+0.4);
}

%Schalter (Schliesser) Vertikal Parameter: xKoordinate, yKoordinate
%Anschluss ist bei (x, y+0.6) & (x, y-0.6)
\newcommand{\tzSV}[2]{
	\draw (#1, #2-0.5) circle(0.1);
	\draw (#1, #2+0.5) circle(0.1);
	
	\draw [thick] (#1, #2+0.4) -- (#1+0.4, #2-0.5);
}

%Horizontale Induktivität Parameter: xKoordinate, yKoordinate, Bezeichnung, Position (a=>above, b=>below)
%Anschluss ist  bei (x-0.7, y) & (x+0.7, y)
\newcommand{\tzLH}[4]{
	\filldraw [black] (#1-0.7, #2-0.25) rectangle (#1+0.7, #2+0.25);
	\IfEqCase{#4}{
		{a}{\node [above] at (#1, #2+0.25) {\Large #3};}%
		{b}{\node [below] at (#1, #2-0.25) {\Large #3};}%
	}
}

%Vertikale Induktivität Parameter: xKoordinate, yKoordinate, Bezeichnung, Position (r=>right, l=>left)
%Anschluss ist  bei (x, y+0.7) & (x, y-0.7)
\newcommand{\tzLV}[4]{
	\filldraw [black] (#1-0.25, #2-0.7) rectangle (#1+0.25, #2+0.7);
	\IfEqCase{#4}{
		{r}{\node [right] at (#1+0.25, #2) {\Large #3};}%
		{l}{\node [left] at (#1-0.25, #2) {\Large #3};}%
	}
}

%Horizontale Induktivität Parameter: xKoordinate, yKoordinate, Bezeichnung, Position (a=>above, b=>below)
%Anschluss ist  bei (x-0.8, y) & (x+0.8, y)
\newcommand{\tzCoilH}[4]{
	\draw (#1-0.8, #2) arc [radius=0.2, start angle=180, end angle = 0];
	\draw (#1-0.4, #2) arc [radius=0.2, start angle=180, end angle = 0];
	\draw (#1, #2) arc [radius=0.2, start angle=180, end angle = 0];
	\draw (#1+0.4, #2) arc [radius=0.2, start angle=180, end angle = 0];
	\IfEqCase{#4}{
		{a}{\node [above] at (#1, #2+0.25) {\Large #3};}%
		{b}{\node [below] at (#1, #2) {\Large #3};}%
	}
}

%Vertikale Induktivität Parameter: xKoordinate, yKoordinate, Bezeichnung, Position (r=>right, l=>left)
%Anschluss ist  bei (x, y+0.7) & (x, y-0.7)
\newcommand{\tzCoilV}[4]{
	\draw (#1, #2+0.8) arc [radius=0.2, start angle=90, end angle = -90];
	\draw (#1, #2+0.4) arc [radius=0.2, start angle=90, end angle = -90];
	\draw (#1, #2) arc [radius=0.2, start angle=90, end angle = -90];
	\draw (#1, #2-0.4) arc [radius=0.2, start angle=90, end angle = -90];
	\IfEqCase{#4}{
		{r}{\node [right] at (#1+0.25, #2) {\Large #3};}%
		{l}{\node [left] at (#1, #2) {\Large #3};}%
	}
}

%Transformator Parameter: xKoordinate, yKoordinate, Bezeichnung
%Anschluss ist  bei (x-1.1, y+0.8) & (x-1.1, y-0.8) &
%					(x+1.1, y+0.8) & (x+1.1, y-0.8)
\newcommand{\tzTransformator}[3]{
	\draw [ultra thick] (#1, #2+0.6) -- (#1, #2-0.6);
	
	\draw (#1-0.6, #2+0.8) arc [radius=0.2, start angle=90, end angle = -90];
	\draw (#1-0.6, #2+0.4) arc [radius=0.2, start angle=90, end angle = -90];
	\draw (#1-0.6, #2) arc [radius=0.2, start angle=90, end angle = -90];
	\draw (#1-0.6, #2-0.4) arc [radius=0.2, start angle=90, end angle = -90];
	
	\draw (#1+0.6, #2+0.8) arc [radius=0.2, start angle=90, end angle = 270];
	\draw (#1+0.6, #2+0.4) arc [radius=0.2, start angle=90, end angle = 270];
	\draw (#1+0.6, #2) arc [radius=0.2, start angle=90, end angle = 270];
	\draw (#1+0.6, #2-0.4) arc [radius=0.2, start angle=90, end angle = 270];
	
	\draw (#1-0.6, #2+0.8) -- (#1-1.1, #2+0.8);
	\draw (#1-0.6, #2-0.8) -- (#1-1.1, #2-0.8);
	\draw (#1+0.6, #2+0.8) -- (#1+1.1, #2+0.8);
	\draw (#1+0.6, #2-0.8) -- (#1+1.1, #2-0.8);
	
	\node [above] at (#1, #2+0.6) {\Large #3};
}

%Strompfeil Parameter: xKoordinate, yKoordinate, Bezeichung, Richtung (r=>right, l=>left, u=>up, d=>down), Position (l=>left, r=>right, a=>above, b=>below)
%Anschluss ist bei
\newcommand{\tzCurrent}[5]{
	\IfEqCase{#4}{
		{r}{\draw [thick, red] (#1-0.2, #2+0.2) -- (#1, #2) --(#1-0.2, #2-0.2);}
		{l}{\draw [thick, red] (#1+0.2, #2+0.2) -- (#1, #2) --(#1+0.2, #2-0.2);}
		{u}{\draw [thick, red] (#1-0.2, #2-0.2) -- (#1, #2) --(#1+0.2, #2-0.2);}
		{d}{\draw [thick, red] (#1-0.2, #2+0.2) -- (#1, #2) --(#1+0.2, #2+0.2);}
	}
	\IfEqCase{#5}{
		{l}{\node [left, red] at (#1-0.2, #2) {\Large #3};}
		{r}{\node [right, red] at (#1+0.2, #2) {\large #3};}
		{a}{\node [above, red] at (#1, #2+0.2) {\Large #3};} 
		{b}{\node [below, red] at (#1, #2-0.2) {\Large #3};}	
	}

}



\input{tikz/tikzSpezial.tex}



\begin{document}
	\setcardpagelayout
	
	\fach{ELT3}
	\kommentar{Repetition ELT2} %Thema

\begin{karte}{Wie gross ist das Elektrische Feld, auf Kopfhöhe 2m) unter einer Hochspannungsleitung?\\
	\small höhe $h = 10m$, Leitungsdurchmesser $d = 3cm$, $U=100kV$}
	\small
	\begin{compactenum}
		\item Ladung berechnen:
		\begin{align*}
			U &= \int_{d/2}^{h} E \cdot dl = \int_{d/2}^{h} \frac{Q}{2 \pi \varepsilon_0 r l} \cdot dl  \\
			&=\frac{Q}{2 \pi \varepsilon_0 l} \cdot \left[ln(r)\right]_{d/2}^{h}\\
			Q & = \frac{U \cdot 2\pi \varepsilon_0\cdot l}{ln(2h/d)}
		\end{align*}
		\item In Formel von $E$ einsetzen. 
		\renewcommand\CancelColor{\color{red}}
		\begin{align*}	
			E &= \frac{Q}{2\pi \varepsilon_0\cdot r \cdot l} =
			\frac{1}{\cancel{2\pi \varepsilon_0}\cdot r \cdot \bcancel{l}} \cdot
			\frac{U \cdot \cancel{2\pi \varepsilon_0}\cdot \bcancel{l}}{ln(2h/d)}\\
			& = \left.\frac{U}{ln(2h/d)\cdot r}\right|_{r=8m} \hspace{-7mm} = 1.92kV/m
		\end{align*}
	\end{compactenum}
\end{karte}
	\kommentar{(Ent-)Laden Kondensator} %Thema
  
\begin{karte}{Wie heisst die Gleichung der folgender Entladefunktion?\\ %Autor: Simon Walker
%Version: 1.0
%Datum: 10.11.2019

\begin{tikzpicture}[xscale=0.8, yscale=0.8]
	%\draw[help lines] (0,0) grid (6,4);
	\normalsize
	\draw [<->] (0, 3.5) -- (0, 0) -- (5.5, 0);
	\node [right] at (5.5, 0) {t};
	\node [above] at (0, 3.5) {u(t)};
	
	\draw (1, -0.1) --  (1, 0.1);
	\node [below] at (1, 0) {$\tau$};
	\draw (2, -0.1) --  (2, 0.1);
	\node [below] at (2, 0) {$2\tau$};
	\draw (3, -0.1) --  (3, 0.1);
	\node [below] at (3, 0) {$3\tau$};
	\draw (4, -0.1) --  (4, 0.1);
	\node [below] at (4, 0) {$4\tau$};
	\draw (5, -0.1) --  (5, 0.1);
	\node [below] at (5, 0) {$5\tau$};
	
	
	\node [left] at (0, 3) {$U_0$};
	\draw (-0.1, 3) --  (0.1, 3);
	\draw[blue, ultra thick, domain=0.01:5] plot (\x, {3*e^(-\x)});
\end{tikzpicture}
}
  	\begin{center}
  		\huge
  		$u(t) = U_0 \cdot e^{-\frac{t}{\tau}} $
  	\end{center}
\end{karte}

\begin{karte}{Wie heisst die Gleichung der folgender Entladefunktion?\\ %Autor: Simon Walker
%Version: 1.0
%Datum: 10.11.2019

\begin{tikzpicture}[xscale=0.8, yscale=0.8]
	%\draw[help lines] (0,0) grid (6,4);
	\normalsize
	\draw [->] (0, 4) -- (0, 4.5);
	\draw [<-] (5.5, 4) --  (0, 4) -- (0, 0);
	\node [right] at (5.5, 4) {t};
	\node [above] at (0, 4.5) {u(t)};
	
	\draw (1, 3.9) --  (1, 4.1);
	\node [above] at (1, 4) {$\tau$};
	\draw (2, 3.9) --  (2, 4.1);
	\node [above] at (2, 4) {$2\tau$};
	\draw (3, 3.9) --  (3, 4.1);
	\node [above] at (3, 4) {$3\tau$};
	\draw (4, 3.9) --  (4, 4.1);
	\node [above] at (4, 4) {$4\tau$};
	\draw (5, 3.9) --  (5, 4.1);
	\node [above] at (5, 4) {$5\tau$};
	
	
	\node [left] at (0, 1) {$U_0$};
	\draw (-0.1, 1) --  (0.1, 1);
	\draw[blue, ultra thick, domain=0.01:5] plot (\x, {(-3)*e^(-\x) + 4});
\end{tikzpicture}
}
	\begin{center}
		\huge
		$u(t) = U_0 \cdot e^{-\frac{t}{\tau}} $\\[10pt]
		\normalsize
		Wobei $U_0$ negativ ist.
	\end{center}
\end{karte}

\begin{karte}{Wie heisst die Gleichung der folgender Ladefunktion?\\ \input{tikz/1_laden.tex}}
	\begin{center}
		\huge
		$u(t) = U_0 \cdot (1 - e^{-\frac{t}{\tau}}) $
	\end{center}
\end{karte}

\begin{karte}{Wie lautet die Differenzialgleichung des Kondensators und was folgt daraus?}
	\begin{center}
		\huge
		$\frac{d u_{C}}{d t}=\frac{i_{C}}{C}$
	\end{center}
	\begin{itemize}
		\item Strom durch einen Kondensator bedingt einer Spannungsänderung
		\item Bei einem grossen Kondensator $(C \gg 0)$ führen auch grosse Ströme nur zu relativ kleinen Spannungsänderungen.
		\item Eine grosse Spannungsänderung an einem Kondensator bedingt einem grossen Strom oder einer kleinen Kapazität
	\end{itemize}
	
\end{karte}

\begin{karte}{Wie kann die Spannung $u_C(t)$ berechnet werden wenn zum Zeitpunkt $t_0$ der Schalter geschlossen wird?\\[10pt] %Autor: Simon Walker
%Version: 1.0
%Datum: 10.11.2019

\begin{tikzpicture}[scale=0.7, every node/.style={scale=0.7}]
%\draw[help lines] (0,0) grid (9, 5);
\normalsize

% Spannungsquelle
\tzUV{0.5}{2.5}{$U_0$}{r}{+}

% Leitungen
%U0 - SW
\draw (0.5, 3) -- (0.5, 4.5) -- (1.4, 4.5);
%SW - R1
\draw (2.6, 4.5) -- (4.5-0.7, 4.5);
%R1 - C
\draw (5.2, 4.5) -- (7.5, 4.5) -- (7.5, 2.75);
\tzKN{6}{4.5} %Knoten 1
%Knoten1 - R_2
\draw (6, 4.5) -- (6, 3.2);
%R2 - Knoten2
\draw (6, 1.8) -- (6, 0.5);
\tzKN{6}{0.5} %Knoten 2
%C - U0
\draw (0.5, 2) -- (0.5, 0.5) -- (7.5, 0.5) -- (7.5, 2.25);

%Schalter (Schliesser)
\tzSH{2}{4.5}

% Widerstand R1
\tzRH{4.5}{4.5}{$R_1$}{b}

% Widerstand R2 (mitte 5.5, 2.5)
\tzRV{6}{2.5}{$R_2$}{l}

% Kondensator C (mitte 7.5, 2.5)
\tzCV{7.5}{2.5}{$C$}{l}	

% Spannungspfeil über Kondensator
\draw [->,thick ,blue] (8.5, 3.5) to [out=-70, in=70] (8.5, 1.5); 
\node [blue, right] at (8.7, 2.5) {\Large $u_C(t)$};


\end{tikzpicture}}
	1. Ersatzspannungsquelle berechnen für den geschlossenen Schalter.\\
	\begin{minipage}{0.6\textwidth}
		%Autor: Simon Walker
%Version: 1.0
%Datum: 10.11.2019

\begin{tikzpicture}[thick,scale=0.7, every node/.style={scale=0.7}]%[xscale=0.6, yscale=0.6]
	
	%\draw[help lines] (0,0) grid (9, 5);
	\normalsize
	
	% Spannungsquelle
	\tzUV{0.5}{2.5}{$U_e$}{r}{+}
	
	% Leitungen
	%Ue - Ri
	\draw (0.5, 3) -- (0.5, 4.5) -- (2.3, 4.5);
	%Ri - C
	\draw (3.7, 4.5) -- (5.5, 4.5) -- (5.5, 2.75);
	%C - Ue
	\draw (0.5, 2) -- (0.5, 0.5) -- (5.5, 0.5) -- (5.5, 2.25); 
	
	% Widerstand R1
	\tzRH{3}{4.5}{$R_i$}{b}
		
	% Kondensator C (mitte 7.5, 2.5)
	\tzCV{5.5}{2.5}{$C$}{l}
	
	% Spannungspfeil über Kondensator
	\draw [->,thick ,blue] (6.5, 3.5) to [out=-70, in=70] (6.5, 1.5); 
	\node [blue, right] at (6.7, 2.5) {\Large $u_C(t)$};
	
	
\end{tikzpicture}

	\end{minipage}
	\begin{minipage}{0.4\textwidth}
		$R_i = \dfrac{R_1\cdot R_2}{R_1+R_2}$\\[5pt]
		$U_e = U_0 \cdot \dfrac{R_2}{R_1 + R_2}$
	\end{minipage}
	2. Spannung $u_C$ berechnen.\\
	$ u_C(t)= U_e \cdot \left(1-e^{\frac{-t}{\tau}}\right)$

\end{karte}

\begin{karte}{Was ist ein Verschiebungsstrom?}
	\begin{center}
		\begin{itemize}
			\item 	Kirchhof ist nicht mehr allgemein gültig. Denn es kann sich nun auch Ladung ansammeln.\\
			\begin{equation*}
			\displaystyle \oint_{\text {Hülle }} \vec{J} \cdot d \vec{s}=0 \quad \Leftrightarrow \quad  \sum_{n} I_{n}=0
			\end{equation*}
			Deshalb muss nun Kirchhof mit dem Verschiebungsstrom erweitert werden:\\
			\begin{equation*}
			\displaystyle \oint_{\text {Hülle }} \vec{J} \cdot d \vec{s}+\frac{d Q_{\text {eingeschlossen }}}{d t}(t)=0
			\end{equation*}
			\item Der Verschiebungsstrom verursacht ebenfalls ein Magnetfeld!
		\end{itemize}
	\end{center}
\end{karte}	



	\kommentar{Induktion} %Thema
  
\begin{karte}{Wie lautet das einfache Induktionsgesetz?}
	\begin{center}
		\huge
		$u_{i}(t)=-\dfrac{d \Phi}{d t}(t)$
	\end{center}
\end{karte}

\begin{karte}{Was sind die Primär- und die Sekundäreffekte der Induktion?}
	\flushleft
	\textbf{Primäreffekt:} Phänomen, dass ein elektrisches \\Feld($\rightarrow$ Spannung) entsteht, wenn sich das Magnetfeld ändert.\\
	\textbf{Sekundäreffekt:} Phänomen, welches eintrifft wenn durch die induzierte Spannung ein Strom fliessen kann. Dieser Strom verursacht ein Gegenfeld nach Lenz.\\[10pt]
	Der Primäreffekt tritt immer auf. Der Sekundäreffekt kann nur auftreten wenn ein Strom fliessen kann.
\end{karte}

\begin{karte}{Was ist elektromagnetische Induktion?}
	Eine induzierte Spannung durch ein zeitlich Änderung des Magnetfelds.\\
	\begin{equation*}
		u_{i}(t)=-\dfrac{d \Phi}{d t}(t)
	\end{equation*}
	\\[10pt]
	Ein Sekundäreffekt tritt dann ein wenn der Stromkreis geschlossen wird. Der fliessende Strom verursacht dann ein Gegenfeld. 
\end{karte}

\begin{karte}{Was ist Ruheinduktion?}
	Die zeitliche Änderung des Magnetfelds kommt von aussen. Die Geometrie in welcher eine Spannung induziert wird ruht.
\end{karte}

\begin{karte}{Wie wird im oder vom Magnetfeld Arbeit verrichtet?}
	Grundsätzlich wird eine Arbeit verrichtet, wenn etwas gegen eine Kraft bewegt wird.
	\begin{enumerate}
		\item Magnetische Flussänderung bewirken Wirbel im elektrischen Feld.
		\item Der Wirbel im elektrischen feld kann einen Strom verursachen. Dieser wiederum verursacht Wirbel im Magnetfeld.
		\item Die Magnetfelder überlagern sich, wirken einander entgegen.4
	\end{enumerate}
\end{karte}

\begin{karte}{Was ist totale Induktion?}
	Die totale Induktion oder Totalinduktion kann sowohl durch eine Bewegung innerhalb eines Magnetfelds, als auch durch zeitlich veränderliche Magnetfelder entstehen – je nach Sichtweise sind die Effekte schlicht ein und dieselben.\\
	Grundsätzlich kann somit gesagt werden, das die Totalinduktion durch die \textbf{Bewegungsinduktion} und die \textbf{Ruheinduktion} zusammengesetzt werden kann.
\end{karte}

\begin{karte}{Was bedeutet Selbstinduktion und Gegeninduktion?}
	Unter Selbstinduktion versteht man die Induktionswirkung eines Stromes auf seinen eigene Geometrie. Im Gegensatz dazu steht die Gegeninduktivität. Sie wird verursacht durch eine Stromänderung einer anderen Spule.
\end{karte}

\begin{karte}{Wie lautet das Ohmsche Gesetz von Induktivitäten?}	
	Der zeitlich veränderliche Strom $i_{L}$ verursacht ein Magnetfeld und damit einen sich ebenfalls zeitlich ändernden magnetischen Fluss $\Phi(t)$. An den Anschlussklemmen entsteht somit die Spannung $u_{L}(t)=\frac{d \Phi}{d t}(t)$.
	Daraus folgt dann das Ohmsche Gesetz von Induktivitäten:
	\begin{center}
		\huge
		$\dfrac{d i_{L}}{d t}(t)=\dfrac{u_{L}(t)}{L}$
	\end{center}
\end{karte}

\begin{karte}{Wie lautet der vollständige Durchflutungssatz?}
	\begin{center}
		$\displaystyle \oint_{C=\partial A} \vec{H} \cdot d \vec{l}=\int_{A}\left(\vec{J}_{\mathrm{frei}}+\frac{d \vec{D}}{d t}\right) \cdot d \vec{s} \quad \stackrel{\circ}{V}_{m}(t)=\Theta(t)+\frac{d \Psi}{d t}(t)$
	\end{center}
	Beispiel: Berechnung der Induktivität einer Ringkernspule\\[10pt]
	\begin{minipage}{0.4\textwidth}
		%Autor: Simon Walker
%Version: 1.0
%Datum: 10.11.2019

\begin{tikzpicture}[xscale=0.7, yscale=0.7, line cap=round, line join=round]
%\draw[help lines] (-2,-2) grid (3,2);
\draw [thick] (-45:1.75 and 2) -- ++(1.75,0) coordinate (a);
\draw [thick] (45:1 and 1.25)++(0,0.25) -- ++(2.25,0) coordinate (b);
\fill (a) circle [radius=.1] (b) circle [radius=.1];

\fill [gray, even odd rule] (0.5,0) 
  ellipse [x radius=1.75, y radius=2] ellipse [x radius=1, y radius=1.25];
\fill [gray] (0,2) rectangle ++(0.5, -0.25) (0,-2) rectangle ++(0.5, 0.25);
\fill [gray!20, even odd rule] 
  ellipse [x radius=1.75, y radius=2] ellipse [x radius=1, y radius=1.25];
\foreach \i in {-45,-22.5,...,45}
  \draw [thick, rounded corners=0.125cm] 
    (\i:1 and 1.25) -- (\i:1.75 and 2) -- +(0.5,0);
 
\draw [blue] ellipse [x radius=1.375, y radius=1.625];
\draw [blue, >-<] (-1.375, -0.11) -- (-1.375, 0.11);
\node [blue, left] at (-1.375, 0) {$l$};
 
\end{tikzpicture}

	\end{minipage}
	\begin{minipage}{0.6\textwidth}
			$L = N \cdot \frac{\Phi}{I}$\\
			\noindent\hspace*{3mm}
			$\Phi = B \cdot A$\\
			\noindent\hspace*{6mm} 
			$B = \mu \cdot H$\\
			\noindent\hspace*{9mm} 
			$H \cdot l = N \cdot I \rightarrow H = \frac{N \cdot I}{l}$\\
			\noindent\hspace*{6mm} 
			$B = \mu \cdot H = \frac{\mu \cdot N \cdot I}{l}$\\
			\noindent\hspace*{3mm}
			$\Phi = B \cdot A = \frac{A\mu \cdot N \cdot I}{l}$\\
			$L = N \cdot \frac{\Phi}{I} = \frac{A\mu \cdot N^2}{l}$
	\end{minipage}
	
\end{karte}

\begin{karte}{Wie wird die Induktivität einer Zylinderspule berechnet?}
	\flushleft
	Das Feld innerhalb einer Zylinderspule kann als Homogen betrachtet werden.\\[5pt]
	\begin{minipage}{0.37\textwidth}
		%Autor: Simon Walker
%Version: 1.0
%Datum: 10.11.2019

\begin{tikzpicture}[
  x=8mm,
  y=cos(30)*8mm,
  z={(0, -sin(30)*8mm)},
]
	\def\cylrad{1}% Zylinder Radius
	\def\cylht{5.5} %Zylinder Höhe
	\def\anzWind{10} %Anzahl Windungen (min 3)
	
	
	
	\draw[ultra thick] %Windungen Hinten
	\foreach \y in {\cylht/(2*\anzWind), 3*\cylht/(2*\anzWind),..., (((2*\anzWind-3)*\cylht)/(2*\anzWind))+0.1} 
	{
	 	plot[smooth, samples=25, variable=\t, domain=0:180]
	 	({-cos(\t)*\cylrad*1.1},
	 	{\y*0.8 + \cylht*0.1 + \t*(\cylht*0.8)/(360*\anzWind)},
	 	{-sin(\t)*\cylrad})
	}

	% Letzte Windung anders
	plot[smooth, samples=12, variable=\t, domain=0:90]
	(({-cos(\t)*\cylrad*1.1},
	{((2*\anzWind-1)*\cylht/(2*\anzWind))*0.8 + \cylht*0.1 + \t*(\cylht*0.8)/(360*\anzWind)},
	{-sin(\t)*\cylrad})
	-- ++(\cylrad*2, 0) coordinate(b);
	
	
	
	%%%%%%%%%%%%%%%%%%%%%%%%%%%%
	%Zylinder Farbig hinterlegen
	%%%%%%%%%%%%%%%%%%%%%%%%%%%%
	\filldraw [gray!25] (-\cylrad, \cylht) -- (-\cylrad, 0) -- (\cylrad, 0) -- (\cylrad, \cylht) -- (-\cylrad, \cylht);
  	%Boden
	\filldraw [gray!25] (0,0) ellipse [x radius=\cylrad, y radius=0.5\cylrad];
	%Deckel
	\filldraw [gray!10] (0,\cylht) ellipse [x radius=\cylrad, y radius=0.5\cylrad];
  
	\draw %Rand Zeichnern
	(-\cylrad, \cylht) -- (-\cylrad, 0) --
    plot[smooth, samples=25, variable=\t, domain=180:360]
      ({cos(\t)*\cylrad}, 0, {-sin(\t)*\cylrad}) --
    (\cylrad, \cylht)
    plot[smooth cycle, samples=51, variable=\t, domain=0:360]
      ({cos(\t)*\cylrad}, \cylht, {-sin(\t)*\cylrad});
	\draw[densely dashed] % Boden Hinten
	plot[smooth, samples=9, variable=\t, domain=0:180]
      ({cos(\t)*\cylrad}, 0, {-sin(\t)*\cylrad});
      
    %%%%%%%%%%%%%%%%%%%%%%
    %Windungen vorne  
    %%%%%%%%%%%%%%%%%%%%%%
	\draw[ultra thick] 
	%Erste Windung anders
	plot[smooth, samples=25, variable=\t, domain=360:270]
	({-cos(\t)*\cylrad*1.1},
	{\cylht*0.1 + (\t-180)*(\cylht*0.8)/(360*\anzWind)},
	{-sin(\t)*\cylrad}) -- ++(\cylrad*2, 0) coordinate(a);
	%Restliche Windungen
	\draw [ultra thick]
    \foreach \y in {\cylht/\anzWind, 2*\cylht/\anzWind,..., (\anzWind-1)*\cylht/\anzWind+0.1} {
      plot[smooth, samples=25, variable=\t, domain=180:360]
        ({-cos(\t)*\cylrad*1.1},
        {\y*0.8 + \cylht*0.1 + (\t-180)*(\cylht*0.8)/(360*\anzWind)},
        {-sin(\t)*\cylrad})
    };
	
	\fill (a) circle [radius=.1] (b) circle [radius=.1];
	
	\draw [thick, blue, <->] (-\cylrad*1.5, 0) -- (-\cylrad*1.5, \cylht);
	\node [blue, left] at (-\cylrad*1.5, \cylht/2) {\large $l$};
	
	\draw [thick, blue, ->] (0, \cylht, 0) -- (\cylrad/1.415, \cylht, -\cylrad/1.415);
	\node [blue, above] at (0, \cylht, 0) {\large $r$};
	
   
\end{tikzpicture}

	\end{minipage}
	\begin{minipage}{0.6\textwidth}
		$L = N \cdot \frac{\Phi}{I}$\\
		\noindent\hspace*{3mm}
		$\Phi = B \cdot A \quad \rightarrow \quad A = r^2\cdot\pi$ \\
		\noindent\hspace*{6mm} 
		$B = \mu \cdot H$\\
		\noindent\hspace*{9mm} 
		$H \cdot l = N \cdot I \rightarrow H = \frac{N \cdot I}{l}$\\
		\noindent\hspace*{6mm} 
		$B = \mu \cdot H = \frac{\mu \cdot N \cdot I}{l}$\\
		\noindent\hspace*{3mm}
		$\Phi = B \cdot A = \frac{A\mu \cdot N \cdot I}{l}$\\[5pt]
		$\boxed{L = \dfrac{A \cdot \mu \cdot N^2}{l}}$
	\end{minipage}
\end{karte}

\begin{karte}{Wie moddeliert man einen idealen Übertrager (Transformator)?}
	\flushleft
	\begin{minipage}{0.45\textwidth}
		%Autor: Simon Walker
%Version: 1.0
%Datum: 10.11.2019

\begin{tikzpicture}[thick,scale=0.7, every node/.style={scale=0.7}]
	%\draw[help lines] (0,0) grid (4, 2);
	
	\tzTransformator{2}{1}{}
	\draw (0, 1-0.8) -- (2-1.1, 1-0.8);
	\draw (0, 1+0.8) -- (2-1.1, 1+0.8);
	\draw (4, 1-0.8) -- (2+1.1, 1-0.8);
	\draw (4, 1+0.8) -- (2+1.1, 1+0.8);
	
	\node[left] at (2-0.6, 1) {\Large $N_1$};
	\node[right] at (2+0.6, 1) {\Large $N_2$};
	
	\tzCurrent{0.5}{1+0.8}{$I_1$}{r}{a}
	\tzCurrent{4-0.5}{1+0.8}{$I_2$}{l}{a}
	
	\draw [->,thick ,blue] (4, 1+0.8-0.1) to [out=-60, in=60] (4, 1-0.8+0.1); 
	\node [blue, left] at (-0.2, 1) {\Large $U_1$};
	
	\draw [->,thick ,blue] (0, 1+0.8-0.1) to [out=-120, in=120] (0, 1-0.8+0.1); 
	\node [blue, right] at (4.2, 1) {\Large $U_2$};
	
\end{tikzpicture}

	\end{minipage}
	\begin{minipage}{0.52\textwidth}
	Ein Idealer Übertrager zeichnet sich dadurch aus, das insbesondere die Eingangsleistung gleich gross ist wie die Ausgangsleistung. $P_{in} = P_{out}$
	\end{minipage}\\[10pt]
	Zudem gelten Folgende Gleichungen:
	\begin{equation*}
		\begin{array}{lcl}
		{\color{red}{I_1}} = \dfrac{\color{red}{I_2}}{ü} & \hspace{15pt} &
		{\color{blue}{U_2}} = \dfrac{{\color{blue}{U_1}}}{ü}\\ \vspace{1pt}\\
		{\color{blue}{U_1}} = {\color{blue}{U_2}} \cdot ü & \hspace{15pt} & 
		\dfrac{{\color{blue}{U_1}}}{{\color{red}{I_1}}} = \dfrac{{\color{blue}{U_2}} \cdot ü}{\color{red}{I_2}}/ü = ü^2\cdot R_L\\
		\end{array}
	\end{equation*}
	
\end{karte}

\begin{karte}{Wie lauteten die Transformatoren Gleichungen?}
	\Large
	\begin{equation*}
		\left[\begin{array}{c} U_1 \\ U_2 \end{array}\right] = 
		\left[ \begin{array}{cc} L_1 & M \\ M & L_2 \end{array} \right]
		\left[ \begin{array}{c} I_1 \\ I_2 \end{array} \right]
	\end{equation*}
	\centering
	%Autor: Simon Walker
%Version: 1.1
%Datum: 06.01.2020

\begin{tikzpicture}[thick,scale=0.9, every node/.style={scale=0.9}]
	%\draw[help lines] (0,0) grid (4, 2);
	
	\tzTransformator{2}{1}{$M$}
	\draw (0, 1-0.8) -- (2-1.1, 1-0.8);
	\draw (0, 1+0.8) -- (2-1.1, 1+0.8);
	\draw (4, 1-0.8) -- (2+1.1, 1-0.8);
	\draw (4, 1+0.8) -- (2+1.1, 1+0.8);
	
	\node[left] at (2-0.6, 1) {\Large $L_1$};
	\node[right] at (2+0.6, 1) {\Large $L_2$};
	
	\tzCurrent{0.5}{1+0.8}{$I_1$}{r}{a}
	\tzCurrent{4-0.5}{1+0.8}{$I_2$}{l}{a}
	
	\draw [->,thick ,blue] (4, 1+0.8-0.1) to [out=-60, in=60] (4, 1-0.8+0.1); 
	\node [blue, left] at (-0.2, 1) {\Large $U_1$};
	
	\draw [->,thick ,blue] (0, 1+0.8-0.1) to [out=-120, in=120] (0, 1-0.8+0.1); 
	\node [blue, right] at (4.2, 1) {\Large $U_2$};
	
\end{tikzpicture}

\end{karte}

\begin{karte}{Was unterscheidet ideale und reale Kopplung?}
	\begin{itemize}
		\item Ideale Kopplung:\\
		$ M = \sqrt{L_1 \cdot L_2} $
		\item Reale Kopplung:\\
		$ M = k \cdot \sqrt{L_1 \cdot L_2} $
		\item Der Kopplungsfaktor $k$ kommt von dem Streufeld. Je mehr Streufeld desto kleiner ist $k$.
	\end{itemize}
\end{karte}

	\kommentar{Wellen \& Strahlungsfelder}  %Thema

\begin{karte}{Was beschreibt das Poynting-Theorem?}
	Grundsätzlich handelt sich beim Poynting-Theorem um Energieerhaltung mit Leistungsübertragung.
	
	\begin{equation*}
	\underbrace{-\frac{d}{d t} \int_{V}\left(w_{e}+w_{m}\right) d v}_{\text{ Zeitliche Änderung der Energie }}
	=
	\underbrace{\int_{V} p d v}_{\text{Leistung}}
	+
	\underbrace{\int_{A=\partial V} \vec{S} \cdot d \vec{s}}_{Leistungsübertragung}
	\end{equation*}
	
	Die Zeitliche Änderung der Energie in einem Volumen, sei sie nun zugeführt oder abgeführt worden, muss gleich gross sein wie die Abgegebene Leistung (Wärme, bei zugeführte Leistung bspw. Solarzelle) plus der austretende oder eintretende Leistung des Volumens.
\end{karte}

\begin{karte}{Was bezeichnet der Poynting Vektor?}
	\begin{equation*}
	\vec{S}=\vec{E} \times \vec{H}
	\end{equation*}
	
	Der Poynting Vektor ist die Leistungsdichte einer Fläche und hat somit die Einheit $[S] = \dfrac{W}{m^2}$.\\
	Der Vektor zeig in die Richtung der Leistung.\\[5pt]
\end{karte}

\begin{karte}{Was besagt die spezielle Relativitätstheorie für die Elektrodynamik?}
	\textbf{Das elektrische und das magnetische Feld sind verschiedene Wirkungsweisen von demselben Phänomen}\\
	Alles kommt auf das Bezugssystem an. Wenn wir uns mit dem Strom mit bewegen sehen wir nur die elektrische Wirkung. Wenn wir uns nicht bewegen, sehen wir die magnetische Wirkung des Stroms.\\
	Wenn wir uns bewegen verhält sich die Zeit anders. Dies wirkt sich bspw. bei der Verkürzung von Längen aus.\\
	Es gibt auch Effekte welche nicht durch die Bewegung beeinflusst werden. z.B. ist die Ladung immer gleich.
\end{karte}

\begin{karte}{Was bedeutet die Polarisation von elektromagnetischen Wellen?}
	Die Polarisation ist die Richtung des Elektrischen Felds.\\
	Die Welle ist x-Polarisiert wenn der E-Feld Vektor in x Richtung zeigt.\\
	%Autor: Simon Walker
%Version: 1.0
%Datum: 29.11.2019
%Lizenz: CC BY-NC-SA

\begin{tikzpicture}[x={(0,1cm)}, y={(-0.6cm,-0.6cm)}, z={(1.2cm, -0.2cm)}]
		
	\def\cycles{1.25} %Anz Schwingungen
	\def\lenght{3}%länge in z richtung
	
	
	\draw [->, thick] (0,0,0)  -- (1.5,0,0) node[above] {\Large$x$};
	\draw [->, thick] (0,0,0) -- (0,1.2,0) node[below left] {\Large$y$};
	\draw [->, thick] (0,0,0) -- (0,0,\lenght+0.5) node[right] {\Large$z$};
	
	\draw[thick, green] %E-Feld Plot
	plot[domain=0:\lenght, samples=200] 
	({1.1*cos(deg(2*\cycles*pi*\x/\lenght))}, 0, \x);
	
	\draw[thick, blue] %M-Feld Plot
	plot[domain=0:\lenght, samples=200] 
	(0, {0.7*cos(deg(2*\cycles*pi*\x/\lenght))}, \x);
	
	\fill[opacity=.1,green!80] %E-Feld füllung
	(0,0,0) --
	plot[domain=0:\lenght, samples=200] 
	({1.1*cos(deg(2*\cycles*pi*\x/\lenght))}, 0, \x)  --
	(0,0,0);
	
	\fill[opacity=.1,blue!80] %M-Feld füllung
	(0,0,0) --
	plot[domain=0:\lenght, samples=200] 
	(0, {0.7*cos(deg(2*\cycles*pi*\x/\lenght))}, \x)  --
	(0,0,0);
	
	\node[green] at (1.3,0,0.3) {\Large$E$}; %Beschriftungen
	\node[blue] at (0,1.1,0.3) {\Large$M$};
	
\end{tikzpicture}

\end{karte}

\begin{karte}{Was ist die Ausbreitungsgeschwindigkeit von elektromagnetischen Wellen?}
	\Large
	\begin{equation*}
		v_{ph} = \frac{1}{\sqrt{\varepsilon \cdot \mu}}
	\end{equation*}	
	 \normalsize
	Somit ist die Ausbreitungsgeschwindigkeit im Vakuum:
	\begin{equation*}
		v_{ph_{Vakuum}} = \frac{1}{\sqrt{\varepsilon_0 \cdot \mu_0}} = c
	\end{equation*}
	
\end{karte}

\begin{karte}{Was versteht man unter einer ebenen Welle?}
	Das magnetische Feld und das elektrische Feld stehen senkrecht zueinander und die Welle ist Harmonisch.\\
\end{karte}

	\kommentar{Wechselstromtechnik}  %Thema

\begin{karte}{Wie lautet das komplexe Signal zur reellen Zeitfunktion\\
	$ u(t) = U_0 \cdot sin(\omega t + \varphi) \rightarrow  \underline{u}(t)$?}
$ u(t) = U_0 \cdot sin(\omega t + \varphi) \rightarrow  \underline{u}(t)$\\
\begin{enumerate}
	\item $ sin $ in $ cos $ umwandeln:\\
	$ u(t)=U_0 \cdot cos(\omega t + \varphi - \color{red}{\pi/2} \normalcolor) $
	\item Imaginärteil hinzufügen:\\
	$ u(t)=U_0 \cdot cos(\omega t + \varphi - \pi/2) + 
	\color{red}{U_0 \cdot j \cdot sin (\omega t + \varphi-\pi/2)}$
	\item Euler:\\
	$ \underline{u}(t) = \underbrace{U_0 \cdot e^{j\omega t} \cdot 
		e^{j\varphi}}_{\text{Phasor}} \cdot \underbrace{e^{-j \pi/2}}_{-j} $
\end{enumerate}
\end{karte}

\begin{karte}{Welche Vorteile bringt uns die komplexe Wechselstromrechnung gegenüber der reellen Wechselstromrechnung?}
	Lineare Netzwerke können durch Multiplikationen gelöst werden anstelle von einer DGL.
\end{karte}

\begin{karte}{Was ist eine Admittanz?}
	\begin{compactitem}
		\item Komplexer Leitwert
		\item Kehrwert der Impedanz \\
		$ \underline{Y} = \dfrac{1}{\underline{Z}} \rightarrow arg(Y)=-arg(Z) \rightarrow \left|\underline{Y}\right| = \dfrac{1}{\left|\underline{Z}\right|} $
		\item Wird benötigt zum berechnen von Parallelen Impedanzen		
	\end{compactitem}
	\vspace{3mm}
	%Autor: Simon Walker
%Version: 1.0
%Datum: 16.12.2019

\begin{tikzpicture}


	\coordinate (A1) at (0,0);
	\coordinate (R1) at (3,0);
	\coordinate (X1) at (3, 1.5);
	
	\draw[red, ->, very thick] (0, 0) -- (3, 0);
	\node[red, below] at (1.5, 0) {$R$};
	
	\draw[red, ->, very thick] (3, 0) -- (3, 1.45);
	\node[red, right] at (3, 0.75) {$X$};
	
	\draw[red, ->, very thick] (0, 0) -- (2.95, 1.5);
	\node[red, above left] at (1.5, 0.75) {$Z$};
	
	\draw[blue, ->, thick] (2.3, 0.5) arc (0:90:0.75);
	\node[blue] at (2, 0.75){$\omega$};
	
	\markangle[blue]{X1}{A1}{R1}{$\textcolor{blue}{\varphi}$}{12};
	
	\coordinate (A2) at (4,0);
	\coordinate (R2) at (6,0);
	\coordinate (X2) at (6, 1);
	
	\draw[red, ->, very thick] (4, 0) -- (6, 0);
	\node[red, below] at (5, 0) {$G$};
	
	\draw[red, ->, very thick] (6, 0) -- (6, 0.95);
	\node[red, right] at (6, 0.5) {$B$};
	
	\draw[red, ->, very thick] (4, 0) -- (5.95, 1);
	\node[red, above left] at (5, 0.5) {$Y$};
	
	\draw[blue, ->, thick] (5.6, 0.4) arc (90:0:0.75);
	\node[blue] at (6.5, 0){$\omega$};
	
	\markangle[blue]{X2}{A2}{R2}{$\textcolor{blue}{\varphi}$}{12};
	
	%\HelpCords{0}{0}{8}{3}
\end{tikzpicture}

\end{karte}

\begin{karte}{Was bezeichnet man als Reaktanz?}
	\begin{itemize}
		\item Die Reaktanz ist der Imaginär Teil der Impedanz. Oder auch der Blindwiderstand.
		\item Das Formelzeichen der Reaktanz ist $X$
		\item Für Kapazitäten ist die Reaktanz negativ
		\item Für Induktivitäten positiv
	\end{itemize}
\end{karte}

\begin{karte}{Wie ist die Phasenverschiebung definiert?}
	Die Referenz ist der Strom.\\
	%Autor: Simon Walker
%Version: 1.0
%Datum: 16.12.2019


\begin{tikzpicture}
	\coordinate (O) at (0, 0);
	\coordinate (I) at (2.5, 0);
	\coordinate (U) at (3, -1);
	\draw[red, ->, very thick] (O) -- (I);
	\node[right, red] at (I) {$\underline{I}$};
	
	\draw[blue, ->, very thick] (O) -- (U);
	\node[right, blue] at (U) {$\underline{U}$};
		
	\draw[thick, ->] ([shift=(0:1.5cm)]O)  arc[start angle=0, end angle=-18.435,radius=1.5cm];
	
	\node at (1.2, -0.2) {$\varphi$};	
\end{tikzpicture}
\\
	Wenn der Winkel negativ ist (wie im Beispiel) sagt man, dass die Spannung nacheilt, oder der Strom voreilt.
\end{karte}

\begin{karte}{Wie lautet das Ohmsche Gesetz einer Kapazität?}
	\begin{itemize}
		\item $\dfrac{du_{C}}{dt}=\dfrac{i_{C}}{C}$
		\item im Komplexen: $\underline{U} = \dfrac{1}{j \omega C} \cdot \underline{I}$
	\end{itemize}
\end{karte}

\begin{karte}{Was ist die Knotenpotentialmethode}
	\begin{itemize}
		\item Systematische Lösungsmethode
		\item Jeder unabhängige Knoten bekommt eine Gleichung für das Potential
		\item $\mathbf{G} \cdot \vec{u} = \vec{i}$\\
		Leitwertmatrix $\cdot$ Spannungsvektor $=$ Stromvektor
		\item im Komplexen 
		$\mathbf{\underline{Y}} \cdot \underline{\vec{U}} = \underline{\vec{I}}$
	\end{itemize}
\end{karte}

\begin{karte}{Wieso macht ein imaginärer Strom überhaupt Sinn?}
	\begin{itemize}
		\item Er Vereinfacht die Rechnung mit Wechselspannungen
		\item Ein rein imaginärer Strom hat eine Phasenverschiebung von $90^\circ$
	\end{itemize}
\end{karte}

\begin{karte}{Was ist eine elektrische Suszeptanz?}
	\begin{itemize}
		\item Blindleitwert
		\item Der Imaginärteil der Admitanz
		\item Formelzeichen $B$
		\item Positives $B$ $\Rightarrow$ Kapazitive Last
		\item \textcolor{red}{\textbf{Achtung:}} Die Suszeptanz ist nicht der Kehrwert der Reaktanz (Blindwiderstand)
	\end{itemize}
\end{karte}

\begin{karte}{Was bezeichnet man als Scheinleistung?}
	\begin{itemize}
		\item Betrag einer Komplexen Leistung
		\item Produkt zwischen Spannung und Strom\\
		$S=\underline{U}\cdot \underline{I}* = P + jQ$
	\end{itemize}
\end{karte}

\begin{karte}{Warum sind Blindströme oftmals unbeliebt?}
	\begin{itemize}
		\item Die Blindströme fliessen durch die Leitungen und verursachen nur Verluste
	\end{itemize}
	
\end{karte}

	
\end{document}