%Autor: Simon Walker
%Version: 1.1
%Datum: 09.11.2019


%Horizontaler Widerstand Parameter: xKoordinate, yKoordinate, Bezeichnung, Position (a=>above, b=>below)
%Anschluss ist  bei (x-0.7, y) & (x+0.7, y)
\newcommand{\tzRH}[4]{
	\draw [thick] (#1-0.7, #2-0.25) rectangle (#1+0.7, #2+0.25);
	\IfEqCase{#4}{
		{a}{\node [above] at (#1, #2+0.25) {\Large #3};}%
		{b}{\node [below] at (#1, #2-0.25) {\Large #3};}%
	}
}

%Vertikaler Widerstand Parameter: xKoordinate, yKoordinate, Bezeichnung, Position (r=>right, l=>left)
%Anschluss ist  bei (x, y+0.7) & (x, y-0.7)
\newcommand{\tzRV}[4]{
	\draw [thick] (#1-0.25, #2-0.7) rectangle (#1+0.25, #2+0.7);
	\IfEqCase{#4}{
		{r}{\node [right] at (#1+0.25, #2) {\Large #3};}%
		{l}{\node [left] at (#1-0.25, #2) {\Large #3};}%
	}
}

%Horizontaler Kondensator Parameter: xKoordinate, yKoordinate, Bezeichnung, Position (a=>above, b=>below)
%Anschluss ist  bei (x-0.25, y) & (x+0.25, y)
\newcommand{\tzCH}[4]{
	\draw [ultra thick] (#1+0.25, #2-0.6) -- (#1+0.25, #2+0.6);
	\draw [ultra thick] (#1-0.25, #2-0.6) -- (#1-0.25, #2+0.6);
	\IfEqCase{#4}{
		{a}{\node [above] at (#1, #2+0.5) {\Large #3};}%
		{b}{\node [below] at (#1, #2-0.5) {\Large #3};}%
	}
}

%Vertikaler Kondensator Parameter: xKoordinate, yKoordinate, Bezeichnung, Position (r=>right, l=>left)
%Anschluss ist  bei (x, y+0.25) & (x, y-0.25)
\newcommand{\tzCV}[4]{
	\draw [ultra thick] (#1-0.6, #2+0.25) -- (#1+0.6, #2+0.25);
	\draw [ultra thick] (#1-0.6, #2-0.25) -- (#1+0.6, #2-0.25);
	\IfEqCase{#4}{
		{r}{\node [right] at (#1+0.5, #2) {\Large #3};}%
		{l}{\node [left] at (#1-0.5, #2) {\Large #3};}%
	}
}

%Vertikale Spannungsquelle Parameter: xKoordinate, yKoordinate, Bezeichnung, Position (r=>right, l=>left), Ausrichtung(+=> + oben, -=> - oben)
%Anschluss ist  bei (x, y+0.5) & (x, y-0.5)
\newcommand{\tzUV}[5]{
	\draw [thick] (#1, #2) circle [radius=0.5];
	\IfEqCase{#5}{
		{+}{\node at (#1, #2+0.25) {\Large $+$};
			\node at (#1, #2-0.25) {\Large $-$};}
		{-}{\node at (#1, #2+0.25) {\Large $-$};
			\node at (#1, #2-0.25) {\Large $+$};}
	}
	
	\IfEqCase{#4}{
		{r}{\node [right] at (#1+0.5, #2) {\Large #3};}%
		{l}{\node [left] at (#1-0.5, #2) {\Large #3};}%
	}
}

%Knoten Parameter: xKoordinate, yKoordinate
\newcommand{\tzKN}[2]{
	\draw[fill=black] (#1, #2) circle(0.07);
}

%Schalter (Schliesser) Horizontal Parameter: xKoordinate, yKoordinate
%Anschluss ist bei (x-0.6, y) & (x+0.6, y)
\newcommand{\tzSH}[2]{
	\draw (#1-0.5, #2) circle(0.1);
	\draw (#1+0.5, #2) circle(0.1);
	
	\draw [thick] (#1-0.4, #2) -- (#1+0.5, #2+0.4);
}

%Schalter (Schliesser) Vertikal Parameter: xKoordinate, yKoordinate
%Anschluss ist bei (x, y+0.6) & (x, y-0.6)
\newcommand{\tzSV}[2]{
	\draw (#1, #2-0.5) circle(0.1);
	\draw (#1, #2+0.5) circle(0.1);
	
	\draw [thick] (#1, #2+0.4) -- (#1+0.4, #2-0.5);
}

%Horizontale Induktivität Parameter: xKoordinate, yKoordinate, Bezeichnung, Position (a=>above, b=>below)
%Anschluss ist  bei (x-0.7, y) & (x+0.7, y)
\newcommand{\tzLH}[4]{
	\filldraw [black] (#1-0.7, #2-0.25) rectangle (#1+0.7, #2+0.25);
	\IfEqCase{#4}{
		{a}{\node [above] at (#1, #2+0.25) {\Large #3};}%
		{b}{\node [below] at (#1, #2-0.25) {\Large #3};}%
	}
}

%Vertikale Induktivität Parameter: xKoordinate, yKoordinate, Bezeichnung, Position (r=>right, l=>left)
%Anschluss ist  bei (x, y+0.7) & (x, y-0.7)
\newcommand{\tzLV}[4]{
	\filldraw [black] (#1-0.25, #2-0.7) rectangle (#1+0.25, #2+0.7);
	\IfEqCase{#4}{
		{r}{\node [right] at (#1+0.25, #2) {\Large #3};}%
		{l}{\node [left] at (#1-0.25, #2) {\Large #3};}%
	}
}

%Horizontale Induktivität Parameter: xKoordinate, yKoordinate, Bezeichnung, Position (a=>above, b=>below)
%Anschluss ist  bei (x-0.8, y) & (x+0.8, y)
\newcommand{\tzCoilH}[4]{
	\draw (#1-0.8, #2) arc [radius=0.2, start angle=180, end angle = 0];
	\draw (#1-0.4, #2) arc [radius=0.2, start angle=180, end angle = 0];
	\draw (#1, #2) arc [radius=0.2, start angle=180, end angle = 0];
	\draw (#1+0.4, #2) arc [radius=0.2, start angle=180, end angle = 0];
	\IfEqCase{#4}{
		{a}{\node [above] at (#1, #2+0.25) {\Large #3};}%
		{b}{\node [below] at (#1, #2) {\Large #3};}%
	}
}

%Vertikale Induktivität Parameter: xKoordinate, yKoordinate, Bezeichnung, Position (r=>right, l=>left)
%Anschluss ist  bei (x, y+0.7) & (x, y-0.7)
\newcommand{\tzCoilV}[4]{
	\draw (#1, #2+0.8) arc [radius=0.2, start angle=90, end angle = -90];
	\draw (#1, #2+0.4) arc [radius=0.2, start angle=90, end angle = -90];
	\draw (#1, #2) arc [radius=0.2, start angle=90, end angle = -90];
	\draw (#1, #2-0.4) arc [radius=0.2, start angle=90, end angle = -90];
	\IfEqCase{#4}{
		{r}{\node [right] at (#1+0.25, #2) {\Large #3};}%
		{l}{\node [left] at (#1, #2) {\Large #3};}%
	}
}

%Transformator Parameter: xKoordinate, yKoordinate, Bezeichnung
%Anschluss ist  bei (x-1.1, y+0.8) & (x-1.1, y-0.8) &
%					(x+1.1, y+0.8) & (x+1.1, y-0.8)
\newcommand{\tzTransformator}[3]{
	\draw [ultra thick] (#1, #2+0.6) -- (#1, #2-0.6);
	
	\draw (#1-0.6, #2+0.8) arc [radius=0.2, start angle=90, end angle = -90];
	\draw (#1-0.6, #2+0.4) arc [radius=0.2, start angle=90, end angle = -90];
	\draw (#1-0.6, #2) arc [radius=0.2, start angle=90, end angle = -90];
	\draw (#1-0.6, #2-0.4) arc [radius=0.2, start angle=90, end angle = -90];
	
	\draw (#1+0.6, #2+0.8) arc [radius=0.2, start angle=90, end angle = 270];
	\draw (#1+0.6, #2+0.4) arc [radius=0.2, start angle=90, end angle = 270];
	\draw (#1+0.6, #2) arc [radius=0.2, start angle=90, end angle = 270];
	\draw (#1+0.6, #2-0.4) arc [radius=0.2, start angle=90, end angle = 270];
	
	\draw (#1-0.6, #2+0.8) -- (#1-1.1, #2+0.8);
	\draw (#1-0.6, #2-0.8) -- (#1-1.1, #2-0.8);
	\draw (#1+0.6, #2+0.8) -- (#1+1.1, #2+0.8);
	\draw (#1+0.6, #2-0.8) -- (#1+1.1, #2-0.8);
	
	\node [above] at (#1, #2+0.6) {\Large #3};
}

%Strompfeil Parameter: xKoordinate, yKoordinate, Bezeichung, Richtung (r=>right, l=>left, u=>up, d=>down), Position (l=>left, r=>right, a=>above, b=>below)
%Anschluss ist bei
\newcommand{\tzCurrent}[5]{
	\IfEqCase{#4}{
		{r}{\draw [thick, red] (#1-0.2, #2+0.2) -- (#1, #2) --(#1-0.2, #2-0.2);}
		{l}{\draw [thick, red] (#1+0.2, #2+0.2) -- (#1, #2) --(#1+0.2, #2-0.2);}
		{u}{\draw [thick, red] (#1-0.2, #2-0.2) -- (#1, #2) --(#1+0.2, #2-0.2);}
		{d}{\draw [thick, red] (#1-0.2, #2+0.2) -- (#1, #2) --(#1+0.2, #2+0.2);}
	}
	\IfEqCase{#5}{
		{l}{\node [left, red] at (#1-0.2, #2) {\Large #3};}
		{r}{\node [right, red] at (#1+0.2, #2) {\large #3};}
		{a}{\node [above, red] at (#1, #2+0.2) {\Large #3};} 
		{b}{\node [below, red] at (#1, #2-0.2) {\Large #3};}	
	}

}


